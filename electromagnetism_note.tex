\documentclass[11pt]{ctexart}
\usepackage{fancyhdr}

% 清除页眉内容并去掉页眉横线
\fancyhf{} % 清空所有页眉页脚
\renewcommand{\headrulewidth}{0pt} % 去掉页眉横线
\pagestyle{fancy} % 应用设置
\usepackage{graphicx}
\usepackage{fix-cm}
\usepackage{setspace}
\usepackage{amsmath}
\usepackage{geometry}
\usepackage{esint}
\linespread{1.5}
\geometry{left=2.5cm,right=2.5cm,top=2.5cm,bottom=2.5cm}
\title{电磁学笔记}
\author{CCY}
\date{\today}
\begin{document}
\tableofcontents%生成目录
\newpage%新建页,内容和目录分开
\maketitle
\section{基本的电磁现象}
\subsection{电荷与电场}
\subsection{导体与电介质}
\subsection{电流场的描述}
\subsection{磁场}
\subsection{磁矩}
\newpage
\section{电场}
\subsection{高斯定理}
\subsection{电场的散度}
\subsection{静电场的电势}
\subsection{静电势能和电场中的能量}
\subsubsection{电荷在外电场的静电势能}
\subsubsection{带点体系的静电能}
电势能存储在电场中:类比弹簧:势能储存在弹簧中,压缩弹簧,能量
被存储到弹簧中;两个正电荷相互靠近,电场发生改变,电势能存储
到电场中;      不同之处:弹簧的弹性势能与两端的质量块无关,
电势能与电场强度有关

静电势能<-->电势<-->做功<-->电场力:
体系的电势能:$E=\frac{1}{2} \int \rho(\vec{r})\phi(\vec{r})dV$
=$\frac{1}{2} \sum q_iU_i$=$\frac{1}{2} \iint \rho_e u \, d\tau $

\noindent 例题:
\begin{enumerate}
    \item 均匀带电球壳带电量为Q,求它的电势能
        
    解:在空间中,将电荷不断从无穷远处移动到球壳上,使球壳的带电量从0增加至Q,在这个过程中,外力做的功即为体系的电势能:

    运用公式求解:$U=\frac{1}{4 \pi \varepsilon_0} \frac{Q}{R}$
    
    $W=\frac{1}{2}Q U =\frac{1}{8 \pi \varepsilon_0} \frac{Q^2}{R}$

    \item 均匀带电球体,半径为$R$,电荷密度为$\rho$,求它的电势能
    
    解:利用公式求解:

    $\vec{E}=\begin{cases}
        \frac{1}{4 \pi \varepsilon_0} \frac{Q}{r^2} & r<R \\
        \frac{1}{4 \pi \varepsilon_0} \frac{Q}{R^3}  r & r>R
    \end{cases}$
   \vspace{10pt}
   \begin{spacing}{2.0}

    $U=\frac{1}{4 \pi \varepsilon_0} (\int_r^R \vec{E} dx+ \int_R ^\infty \vec{E} dx)
    =\frac{1}{4 \pi \varepsilon_0 R^3} \int_r^R x dx+ \int_R ^\infty \frac{R^3}{x^2} dx
    =\frac{Q}{4 \pi \varepsilon_0 R^3}(-\frac{x}{R^3} \mid_R^\infty + \frac{1}{2} x^2\mid_r^R)
    =\frac{Q}{4 \pi \varepsilon_0 R ^3} \frac{3R^2-r^2}{2}$
    
    $ E_\phi =\frac{1}{2} \int \rho u d\tau$

    $= \frac{1}{4} \frac{Q}{4 \pi \varepsilon_0 R^3} \frac{3Q}{4 \pi R^3} \int_{\Omega} (3R^2-r^2)dV$

   \end{spacing}
   \vspace{10pt}
    $=\frac{3 Q^2}{20 \pi \varepsilon_0 R }$
    

\end{enumerate}
\subsubsection{电场的能量}
在某一带电球壳周围存在伴存电场,假设其等势面是一系列封闭曲面,空间被这些等势面划分成许多薄层,如图所示:

将带点球壳每一部分从无穷远处移动到原处,在移动dx段距离的过程中,只有这一部分的电场发生了变化,故外力所做的功存储在这一部分空间中
两等势面的电势差为$dU$,每一薄层的电能$dW=\frac{1}{2} QdU$,整个电场的能量$W=\frac{1}{2}Q \int dU=\frac{1}{2}QU_0$

根据高斯定理有:$\oint \vec{E}dS=\frac{1}{\varepsilon_0} Q$

设$dl$和$dS$的方向一致(即指向S的外法线方向)则有:$dU=\vec{E} dl$

所以:$W=\oint \frac{1}{2}\varepsilon_0 \vec{E} \cdot \vec{E}d\tau$

所以单位体积内的电场能量为:
\begin{equation}
w_e=\frac{1}{2} \varepsilon_0 E^2
\end{equation}

$w_e$叫做电场的能量密度,虽然是由孤立带点等势面的特例得到,却对所有电场都适用

例题:

\begin{itemize}
\item 有一个厚度为t的球壳,内半径为R,将无穷远处的一个点电荷移动到导体空腔圆心处,求在移动过程中外力所做的功

解:

方法一:

    从初态到末态,可以看作只有导体所占空间处的电场消失了,用电场的能量密度求解:

$W=\frac{1}{2} \varepsilon_0 \int_R^{R+t}\vec{E}^2 d\tau 
=\frac{Q^2}{32 \pi^2 \varepsilon_0} \int_R^{R+t}\frac{1}{r^4} d\tau 
=\frac{Q^2}{8 \pi \varepsilon_0}(\frac{1}{R+t}-\frac{1}{R})$

方法二:
    求移动后体系的静电能

\end{itemize}
\subsubsection{电子的经典半径}
带电粒子和他的电场是不可分割的整体,若电场的能量为W,按狭义相对论粒子的质量为
\begin{equation}
m_{em}=\frac{W}{c^2}
\end{equation}
其中,$m_{em}$是粒子的电磁质量,$W$是电子的电场能量,$c$是光速。
$W$与电荷的分布有关,

假设粒子半径为$r_0$,电荷密度为$\rho$,电荷量为$q$,

电荷均匀分布在球面上:
\begin{equation}
    m_{em}=\frac{1}{2} \frac{Q^2}{4 \pi \varepsilon_0 c^2 R}
\end{equation}

电荷均匀分布在球体上:
\begin{equation}
    m_{em}=\frac{3}{5} \frac{Q^2}{4 \pi \varepsilon_0 c^2 R}
\end{equation}

迄今还无法测知电荷的分布,作为一种估计,$m_{em}=\frac{Q^2}{4 \pi \epsilon_0 c^2 R}
$
由上式可得:
\begin{equation}
    R=\frac{Q^2}{4 \pi \varepsilon_0 c^2 m_{em}}
\end{equation}
粒子的电量是可以测得的,但是$m_{em}$是未知的

电子的电荷量$q=-1.6 \times 10^{-19} C$,质量$m_e=9.1 \times 10^{-31} kg$,
若将其全部估算为电磁质量,则
\begin{equation}
    R=\frac{(1.6 \times 10^{-19})^2}{4 \pi \varepsilon_0 c^2 m_e} \approx 2.82 \times 10^{-13} m
\end{equation}
即电子的经典半径,是根据宏观电磁学对电子半径的估计,不可能是准确的。

\subsection{导体的静电平衡和电子的发射}
\subsubsection{导体中的自由电子}
势阱: 在导体中,电荷分布是均匀的,电场强度为0,电势是常数,

逸出功:

\subsubsection{静电平衡}
均匀导体板在匀强电场中的情况

不规则导体在电场中的情况

导体有空腔的情况:空腔内部电场为0

初始外电场为0,导体有空腔,空腔内有电荷:

同上,但导体接地:

同上,在导体外部有电荷:

\section{电流场}
\subsection{导体中的传导电流}
\subsubsection{欧姆定律}
电流密度:在导体中任取一个小面积元$dS$,单位法向量为$\vec{n}$,电流密度为$\vec{j}$,
\begin{equation}
    dI=\vec{j} \cdot \vec{n} dS
    =jdS \cos \theta
\end{equation}
单位面积的电流密度为$\vec{j}=nq\vec{v}$,其中,$n$为单位体积内的自由电子数,$q$为电子的电荷量,$\vec{v}$为电子的漂移速度

真实速度:电子在导体中做无规则运动,常温下平均速率约为$10^5~10^6m/s$平均速度为0不形成电流

漂移速度:由于电场的作用,电子有逆着电场方向的加速度和速度 可用$\vec{u}$表示,数量级约为$10^{-4}~~10^{-5}m/s$,正是漂移速度产生了宏观电流

在导体中,电子受到电场力的作用,对电子使用冲量定理:
\begin{equation}
    \vec{v_0}=0;
    m\vec{u_末}=-eE
\end{equation}
取一个
\subsection{电源及其电动势}
\subsection{电容}
\subsection{电流场的连续性}



\newpage

\section{磁场与电磁效应}
\subsection{磁场的通量和环量}
\subsubsection{磁场的通量————磁场的高斯定理}
在1.4.4中讲了毕奥-萨伐尔定律:
在相对电流元矢径为$\vec{r}$的点P处,电流的伴存磁场为:
\begin{equation}
    dB=\frac{\mu_0}{4 \pi} \frac{I d\vec{l} \times \vec{r}}{r^3}
\end{equation}

对任意曲面S,磁感应强度矢量的磁通量:

\begin{equation}
    \phi_B=\oiint_S \vec{B} \cdot d\vec{S}
\end{equation}

由于磁感线是闭合曲线,所以对任意闭合曲面S,磁通量为零,即$\oiint_S d\vec{B} \cdot d\vec{S} =0$

对闭合曲面,根据高斯定理:$ \oiint _S \vec{B} \cdot d\vec{S}=\iiint_V \nabla \cdot \vec{B} dV =0 $,
因为S是任意的,所以必然有:
\begin{equation}
    \nabla \cdot \vec{B} =0
\end{equation}

说明磁场的散度为0.散度为0的场叫做无散场,电流场也是无散场。


\subsubsection{磁场的环量安培环路定理}
安培环路定理:磁感应强度沿任一有向闭合环路L的环量等于穿过该环路所围成的任意面S的电流代数和的$\mu_0$倍:

\begin{equation}
    \oint_L \vec{B} \cdot d\vec{r}=\mu_0 \Sigma_{i=1}^n I_i
\end{equation}

鉴于全电流的概念:
\begin{equation}
    \Sigma_{i=1}^n I_i=\iint_S \vec{j} \cdot d\vec{S}
\end{equation}

所以安培环路定理可以写成:
\begin{equation}
    \oint_L \vec{B} \cdot d\vec{r}=\mu_0 \iint_S \vec{j} \cdot d\vec{S}
\end{equation}

根据斯托克定律:
\begin{equation}
    \oint_L \vec{B} \cdot d\vec{r}=\iint_S (\nabla \times \vec{B}) \cdot d\vec{S}
\end{equation}
其中,$\nabla =(\partial_x,\partial_y,\partial_z)$,$\nabla \times \vec{B} $是磁场的旋度

则,

\begin{equation}
    \iint_S \nabla \times \vec{B} \cdot d\vec{S}=\mu_0 \iint_S \vec{j} \cdot d|vec{S}
\end{equation}

所以,
\begin{equation}
    \nabla \times \vec{B}=\mu_0 \vec{j}
\end{equation}
这是安培环路定律的微分形式

\subsubsection{面电流的磁场}
面电流:若电流分布在厚度为$\varDelta d$的薄层中,且$\varDelta d$可视为无穷小量,则认为电流分布在一个面上,称为面电流

设一个薄层厚度为$\varDelta d$,宽度为$\varDelta l$,电流为$\varDelta I$,因为$\varDelta d$是无穷小量,用$\frac{\Delta I}{\Delta d}$表示电流密度,称为面电流密度,记为$\vec{j_m}$,单位为$A/m$
\begin{equation}
    j_m=\frac{\Delta I}{\Delta d} 
\end{equation}

若通过薄层的电流密度矢量为$\vec{j}$,则有:
\begin{equation}
    \varDelta I=\vec{j} \varDelta L \varDelta d
\end{equation}

则:$\vec{j_m}=\vec{j} \varDelta d$

在实际问题中,只有薄层的厚度$\varDelta d$可视为无穷小量时,才能用面电流的概念。若$j_m$不是无穷小量,则j必然为无穷大量,此时4.1.2最后的式子(微分形式的安培环路定理)将不再适用,但是积分形式的安培环路定理仍然适用。

\subsubsection{柱电流面的磁场}

有一截面均匀的无限长圆柱形电流面通有均匀稳定的横向面电流,电流面密度为$j_m$,由于磁感应强度与电流元垂直,所以$\vec{B}$只有z分量。又因为$\nabla \times \vec{B}=0$
\subsection{通电螺线管的磁场}

\subsection{电磁感应定律与涡旋电场}

\subsection{几种电磁感应现象}

\subsection{磁场的能量}

\subsection{电路中的电磁感应}





\section{电介质}

\subsection{电介质的极化}
根据电荷能否在物质中在宏观上(即离开电子原来所在的原子,在固体物理上叫做电子公有化)自由移动,将物质分为导体和绝缘体,绝缘体又称介质。但这种分类并不绝对,如半导体既是导体又是绝缘体;“击穿”现象:绝缘体变成了导体。

电介质:能对外加电场做出相应的介质


\subsubsection{介质极化的微观原理}
在外加电场$\vec{E_0}$的作用下,介质中产生极化电荷$Q^\prime$,极化电荷产生电场$\vec{E^\prime}$,空间中的电场变为$\vec{E}=\vec{E_0}+\vec{E^\prime}$,这就是电介质极化的微观过程


将每一个分子都看成一个电偶极子(在空间中产生电场;在外加电场的作用下会受力),$\vec{p}=q\cdot\vec{L}$.

无极分子介质的位移极化:氢气($H_2$),二氧化碳($CO_2$)等,正负电荷中心重合,无极性。在外加电场的作用下,正负电荷中心发生位移,形成了电偶极子,p约为$10^{-29}-10^{-30}cm$量级

有极分子的取向极化:例如氯化氢($HCl$)。正负电荷中心不重合,有电偶极矩。无外电场的情况下,由于热运动,分子的取向杂乱无章,所有分子的整体在宏观上无电偶极矩。外加电场之后,所有的电偶极矩受到力矩的作用,使电偶极子尽量往电场的方向分布,最终与热运动形成了一种平衡,宏观上有电偶极矩。且电偶极矩大小与温度有关。

介质的极化主要是有极分子的取向极化。

\subsubsection{极化强度矢量}
\begin{equation}
    \vec{P}=\frac{\Sigma \vec{p}_i}{dV}
\end{equation}
单位:$C/m^2$,是空间位置的函数,只存在于介质内部

\subsubsection{极化电荷与极化强度矢量之间的关系}
在外加电场的作用下,在空间中的介质中取一个闭合曲面,那么介质中的极化电荷与极化强度之间的关系是什么?

在介质中的闭合曲面中,由于电荷极化时发生的位移非常小,可视为曲面内外部极化电荷为0,仅在闭合曲面表面的薄层中产生的极化电荷(电偶极子)对曲面包围体积中的极化电荷有贡献。取闭合曲面的一个面积微元$dS$,单位法向量为$\vec{n}$,取高为$l$,以$dS$为底面的一个小圆柱体,

\begin{equation}
    dV=l dS |\cos \theta|
\end{equation}
\begin{equation}
    dN=n dV=n l|\cos\theta| dS
\end{equation}
\begin{equation}
    dQ^\prime=q n l |\cos \theta| dS
    =-\vec{P}\cdot d\vec{S}
\end{equation}
虽然介质的均匀极化非常困难,但在该微元中,可视为均匀极化,各电偶极矩相等。则由定义知:$\vec{P}=n q \vec{l}$,所以:
\begin{equation}
    Q^ \prime=-\oint_S \vec{P} \cdot d\vec{S}
\end{equation}
两边同时对体积求微分:
\begin{equation}
\rho^\prime (\vec{r})=-\nabla \cdot \vec{P}(\vec{r})
\end{equation}





\end{document}
